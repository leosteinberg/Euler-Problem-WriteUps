% !TeX document-id = {a5f80066-b090-4a04-b7ab-c86d337b294a}
\documentclass[12pt]{article}

\usepackage{array}
\usepackage{amsfonts}
\usepackage{amsmath}
\usepackage{amssymb}
\usepackage{booktabs}
\usepackage{circuitikz}
\usepackage{csvsimple}
\usepackage{enumitem}
\usepackage{fancyhdr}
%%\usepackage{fontawesome}
\usepackage{hyperref}
\usepackage{imakeidx}
%%\usepackage{inputenc}
\usepackage[nopatch]{microtype}
\usepackage{minted}
\usepackage{siunitx}
\usepackage{tikz}
\usepackage{xcolor}
\usepackage{geometry}
\geometry{left=1in}
	

\title{Euler Problem 1 Writeup}
\author{Leo Steinberg}


\begin{document}




\maketitle


\section{Problem Statement:}
If we list all the natural numbers below 10 that are multiples of 3 or 5, we get 3, 5, 6 and 9. The sum of these multiples is 23.
\\
Find the sum of all the multiples of 3 or 5 below 1000.
\subsection{Coding adapation}
Haker Rank provides the following code adapation: \\

If we list all the natural numbers below $10$ that are multiples of $3$ or $5$ , We get $3, 5, 6, 9$. The summ of theese multiples is $23$. Find the sum of all the multiples of $3$ or $5$ below $N$.
\subsubsection*{Input Format}
First line contains $T$ that denotes the number of test cases. This is followed by $T$ lines, each containing an integer, $N$.
\subsubsection*{Constraints}
\begin{itemize}
\item $1 \leq T \leq 10^{5} $
\item $1 \leq N \leq 10^{9} $
 \end{itemize}

\subsubsection*{Output Format}
For each test case, print an integer that denotes the sum of all the multiples of $3$ or $5$ below $N$.

\subsection{Solution}

\subsubsection{bash}
My first atempt was simply to bash this. There are quicker ways and it could all be done in main, but I found this to be the most "human readable" way of doing it.

\begin{minted}[frame=single,framesep=6pt]{C}
	
	#include <math.h>
	#include <stdio.h>
	
	long SumUnderN(long);
	
	int main(){
		int i; //counter
		int T; // # of testcases
		long N; // place to store each instance of N.
		
		scanf("%d",&T);
		
		for (i = 0;i < T;i++)
		{
			scanf("%ld",&N); 
			printf("%ld\n",SumUnderN(N));
		}
		return 0;
	}
	long SumUnderN(long N)
	{
		long i; //counter
		long sum = 0; // sum
		for(i = 1; i< N;i++)
		{
			sum += i * ((i % 3 == 0) || (i % 5 == 0));
		}
		return sum;
	}
	
\end{minted}
It should be noted the long data type needs to be used as N can be up to $10^{9}$.
\subsubsection{Optimization / Math}
The problem with this method is it's inefficient. For most problems like this with modern compuational power this is not a problem, however for  the sake of good practice we should do better. As such Hackerank fails two of the automated testcases as it is also testing for this. 
\\
 The multiples of $3$ and $5$ are by definition at regular intervals.  We could edit our for loops to ittrate twice through the given number $N$, in steps of $3$ then $5$, but this still isn't optimal. Some basic Number Theory and inclusion/exclusion tell us this can be calulated for a given number with division. If we add up all the mutiples of 3, then the mutplies of 5 then subtract the multiples of 15 we get our answer Thus we get:
 \[
   \text{answer}  =
    \left(\sum_{k = 1}^{\lfloor \frac{N}{3} \rfloor}  \left( 3\cdot k\right) \right)  +
        \left(\sum_{k = 1}^{\lfloor \frac{N}{5} \rfloor}  \left( 5\cdot k\right) \right)  -
            \left(\sum_{k = 1}^{\lfloor \frac{N}{15} \rfloor}  \left( 15\cdot k\right) \right)  
 	\]
 \[
 	=  3\cdot  \frac{\lfloor \frac{N}{3} \rfloor \left( \lfloor \frac{N}{3} \rfloor + 1\right)  }{2} +
 	5\cdot \frac{\lfloor \frac{N}{5} \rfloor \left( \lfloor \frac{N}{5} \rfloor + 1\right)  }{2} -
 	15\cdot \frac{\lfloor \frac{N}{15} \rfloor \left( \lfloor \frac{N}{15} \rfloor + 1\right)  }{2}
 \]
We subtract the sum of multiples up to  $\lfloor \frac{N}{15} \rfloor$ to remove the double counting of numbers that are both multiples of $3$ and $5$. We can implement this in code by changing our SumUnderN function using some custom functions from the math.h header file.
\\
\begin{minted}[frame=single,framesep=6pt]{C}
	long SumUnderN(long N)
	{
		long M3 = floor(((N-1)/3)); // multiples of 3
		long M5 = floor(((N-1)/5));// multiples of 5
		long M15 = floor(((N-1)/15));// multiples of 15
		long sum = (3*(M3*(M3+1))/2) + 
		(5*(M5*(M5+1))/2) - 
		(15*(M15*(M15+1))/2);
		return sum;
	}
\end{minted}

\end{document}


%% needed for minted color
% !TeX TXS-program:compile = txs:///pdflatex/[--shell-escape] 
